Our approach to adaptive refinement in \iamr\ uses a nested
hierarchy of logically-rectangular grids with simultaneous refinement
of the grids in both space and time.  The integration algorithm on the grid hierarchy
is a recursive procedure in which coarse grids are advanced in time,
fine grids are advanced multiple steps to reach the same time
as the coarse grids and the data at different levels are then synchronized.

During the regridding step, increasingly finer grids
are recursively embedded in coarse grids until the solution is
sufficiently resolved.  An error estimation procedure based on
user-specified criteria (described in Section \ref{sec:tagging}) 
evaluates where additional refinement is needed
and grid generation procedures dynamically create or
remove rectangular fine grid patches as resolution requirements change.

A good introduction to the style of AMR used here is in Lecture 1
of the Adaptive Mesh Refinement Short Course at
https://ccse.lbl.gov/people/jbb/index.html

\section{Tagging for Refinement}
\label{sec:tagging}

The dynamic creation and destruction of grid levels is a fundamental part of \iamr\ capabilities.
At regular intervals (set by the user), each Amr level that is not the finest allowed for the run
will invoke a ``regrid'' operation.  When invoked, a list of error tagging functions is traversed. For each,
a field specific to that function is derived from the state over the level, and passed through a kernel
that ``set'''s or ``clear'''s a flag on each cell.  The field and function for each error tagging quantity is
identified in the setup phase of the code where the state descriptors are defined (i.e., in {\tt NS\_setup.cpp}).
Each function in the list adds or removes to the list of cells tagged for refinement. This final list of tagged
cells is sent to a grid generation routine, which uses the Berger-Rigoutsos algorithm to create rectangular grids
which will define a new finer level (or set of levels).  State data is filled over these new grids, copying where
possible, and interpolating from coarser level when no fine data is available.  Once this process is complete,
the existing Amr level(s) is removed, the new one is inserted into the hierarchy, and the time integration
continues.

The traditional approach to setting up and controlling the regrid process involves explicitly
creating (``hard coding'') a number of functions directly -- that is discussed below in section~\ref{sec:oldtagging}.
\iamr\ also provides a limited capability to augment
the standard set of error functions that is based entirely on runtime data specified in the inputs ({\tt ParmParse})
data.  The following example portion of a ParmParse'd input file demonstrates the usage of this feature:

\begin{verbatim}
      amr.refinement_indicators = hi_rho lo_temp gradT

      amr.high_rho.max_level = 3
      amr.hi_rho.value_greater = 1.
      amr.hi_rho.field_name = density

      amr.lo_temp.max_level = 1
      amr.lo_temp.value_less = 450
      amr.lo_temp.field_name = temp

      amr.gradT.max_level = 2
      amr.gradT.adjacent_difference_greater = 20
      amr.gradT.field_name = temp
      amr.gradT.start_time = 0.001
      amr.gradT.end_name = 0.002
\end{verbatim}

Here, we have added three new custom-named criteria -- {\tt hi\_rho}: cells with the desnity than 1;
{\tt lo\_temp}: cells with T less than 450K, and {\tt gradT}: cells having a temperature difference of 20K from that of their
immediate neighbor.  The first will trigger up to Amr level 3, the second only to level 1, and the third to level 2.
The third will be active only when the problem time is between 0.001 and 0.002 seconds.

Note that these additional user-created criteria operate in addition to those defined as defaults.  Also note that
these can be modified between restarts of the code.  By default, the new criteria will take effect at the next
scheduled regrid operation.  Alternatively, the user may restart with {\tt amr.regrid\_on\_restart = 1} in order to
do a full (all-levels) regrid after reading the checkpoint data and before advancing any cells.


\section{Older style refinement tagging}
\label{sec:oldtagging}

\iamr\ also determines what zones should be tagged for refinement at the next 
regridding step using a set of built-in routines that test on 
quantities such as the density and pressure and determining whether the 
quantities themselves or their gradients pass a user-specified threshold. 
This may then be extended if {\tt amr.n\_error\_buf} $> 0$ to a certain number 
of zones beyond these tagged zones. This section describes the process 
by which zones are tagged, and describes how to add customized tagging criteria.

The routines for tagging cells are located in the {\tt Prob\_nd.f90} file 
in the {\tt Src\_nd} directory. The main routines are 
{\tt denerror}, {\tt temperror}, {\tt presserror}, {\tt velerror}. They refine
based on density, temperature, pressure, and velocity, respectively. The same approach is used 
for all of them. As an example, we consider the density tagging routine. There 
are four parameters that control tagging. If the density in a zone is greater than 
the user-specified parameter {\tt denerr}, then that zone will be tagged 
for refinement, but only if the current AMR level is less than the 
user-specified parameter {\tt max\_denerr\_lev}. Similarly, 
if the absolute density gradient between a zone and any adjacent zone 
is greater than the user-specified parameter {\tt dengrad}, that zone 
will be tagged for refinement, but only if we are currently on a level 
below {\tt max\_dengrad\_lev}. Note that setting {\tt denerr} alone will 
not do anything; you'll need to set {\tt max\_dengrad\_lev} $>= 1$ for
this to have any effect.

All four of these parameters are set in the {\tt \&tagging} namelist 
in your {\tt probin} file. If left unmodified, they default to a value 
that means we will never tag. The complete set of parameters that can 
be controlled this way is the following:
\begin{multicols}{5}
\begin{itemize}
  \setlength{\itemindent}{-3em}
  \item {\tt denerr}
  \item {\tt max\_denerr\_lev}
  \item {\tt dengrad}
  \item {\tt max\_dengrad\_lev}
  \item {\tt temperr}
  \item {\tt max\_temperr\_lev}
  \item {\tt tempgrad}
  \item {\tt max\_tempgrad\_lev}
  \item {\tt velerr}
  \item {\tt max\_velerr\_lev}
  \item {\tt velgrad}
  \item {\tt max\_velgrad\_lev}
  \item {\tt presserr}
  \item {\tt max\_presserr\_lev}
  \item {\tt pressgrad}
  \item {\tt max\_pressgrad\_lev}
  \item {\tt raderr}
  \item {\tt max\_raderr\_lev}
  \item {\tt radgrad}
  \item {\tt max\_radgrad\_lev}
\end{itemize}
\end{multicols}
Since there are multiple algorithms for determining 
whether a zone is tagged or not, it is worthwhile to specify 
in detail what is happening to a zone in the code during this step. 
We show this in the following pseudocode section. A zone 
is tagged if the variable {\tt itag = SET}, and is not tagged 
if {\tt itag = CLEAR} (these are mapped to 1 and 0, respectively).

\begin{verbatim}
itag = CLEAR

for errfunc[k] from k = 1 ... N
    // Three possibilities for itag: SET or CLEAR or remaining unchanged
    call errfunc[k](itag)  
end for
\end{verbatim}

In particular, notice that there is an order dependence of this operation; if {\tt errfunc[2]} 
{\tt CLEAR}s a zone and then {\tt errfunc[3]} {\tt SET}s that zone, the final operation will 
be to tag that zone (and vice versa). In practice by default this does not matter, because the 
built-in tagging routines never explicitly perform a \texttt{CLEAR}. However, 
it is possible to overwrite the {\tt Tagging\_nd.f90} file if you want to change how 
{\tt ca\_denerror, ca\_temperror}, etc. operate. This is not recommended, and if you do so
be aware that {\tt CLEAR}ing a zone this way may not have the desired effect.

We provide also the ability for the user to define their own tagging criteria.
This is done through the Fortran function {\tt set\_problem\_tags} in the 
{\tt problem\_tagging\_*d.f90} files. This function is provided the entire 
state (including density, temperature, velocity, etc.) and the array 
of tagging status for every zone. As an example of how to use this, suppose we 
have a 3D Cartesian simulation where we want to tag any zone that has a 
density gradient greater than 10, but we don't care about any regions 
outside a radius $r > 75$ from the problem origin; we leave them always unrefined. 
We also want to ensure that the region $r \leq 10$ is always refined.
In our {\tt probin} file we would set {\tt denerr = 10} and {\tt max\_denerr\_lev = 1}
in the {\tt \&tagging} namelist. We would also make a copy of 
{\tt problem\_tagging\_3d.f90} to our work directory and set it up as follows:
\clearpage
\begin{verbatim}
subroutine set_problem_tags(tag,tagl1,tagl2,tagl3,tagh1,tagh2,tagh3, &
                            state,state_l1,state_l2,state_l3,state_h1,state_h2,state_h3,&
                            set,clear,&
                            lo,hi,&
                            dx,problo,time,level)

  use bl_constants_module, only: ZERO, HALF
  use prob_params_module, only: center
  use meth_params_module, only: URHO, UMX, UMY, UMZ, UEDEN, NVAR
 
  implicit none
  
  integer         ,intent(in   ) :: lo(3),hi(3)
  integer         ,intent(in   ) :: state_l1,state_l2,state_l3, &
                                    state_h1,state_h2,state_h3
  integer         ,intent(in   ) :: tagl1,tagl2,tagl3,tagh1,tagh2,tagh3
  double precision,intent(in   ) :: state(state_l1:state_h1, &
                                          state_l2:state_h2, &
                                          state_l3:state_h3,NVAR)
  integer         ,intent(inout) :: tag(tagl1:tagh1,tagl2:tagh2,tagl3:tagh3)
  double precision,intent(in   ) :: problo(3),dx(3),time
  integer         ,intent(in   ) :: level,set,clear

  double precision :: x, y, z, r

  do k = lo(3), hi(3)
     z = problo(3) + (dble(k) + HALF) * dx(3) - center(3)
     do j = lo(2), hi(2)
        y = problo(2) + (dble(j) + HALF) * dx(2) - center(2)
        do i = lo(1), hi(1)
           x = problo(1) + (dble(i) + HALF) * dx(1) - center(2)

           r = (x**2 + y**2 + z**2)**(HALF)

           if (r > 75.0) then
             tag(i,j,k) = clear
           elseif (r <= 10.0) then
             tag(i,j,k) = set
           endif
        enddo
     enddo
  enddo
  
end subroutine set_problem_tags
\end{verbatim}

